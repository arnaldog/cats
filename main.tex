\documentclass[12pt]{article}

\usepackage[utf8]{inputenc}
\usepackage[spanish]{babel}

\usepackage{amsmath}
\usepackage{amssymb}

%%% MATH COMMANDS SHORCUTS %%%%
\newcommand{\R}{\ensuremath{\mathbb{R}}}
\newcommand{\Z}{\mathbb{Z}}
\newcommand{\N}{\mathbb{N}}
\newcommand{\C}{\mathbb{C}}
\newcommand{\txt}[1]{\text{#1}}

\newcommand{\D}[2]{\frac{d#1}{\d#2}} 

%%% TEXT SHORCUT COMMANDS %%%
\newcommand{\italic}[1]{\textit{#1}}


\title{Consideraciones y Definiciones Matemáticas}

\begin{document}
\maketitle

% EL TIEMPO
Se puede definir al tiempo como un continuo $t \in \R+$ ó $\R$, donde $t = h(u), u \in
\R$, donde $$\frac{dh(u)}{du} > 0, \quad \forall u \in R$$. El tiempo $t \in
\R_+$ podría especificarse como $\R_+ - \{0\}$ suponiendo como desconocido el
tiempo cero o \textit{original}.

Al respecto, la función $h(u)$ es estrictamente creciente. Al respecto, se puede
considerar a la velocidad del tiempo como la rapidez de cambio de la 
$$
	h(u) = \int_{\epsilon}^{u} \frac{f'(s)}{f(s)}ds = \log f(s) 
$$

Por lo tanto, podríamos considerar que el tiempo $t$ correponde a una percepción
logaritmica de un fenómeno secuencial. 

% MOTION AND PHASE STATE
El movimiento está asociado a un conjunto $M \in \R$ llamado \textit{espacio de
fase}.
En la mecánica clásica, el espacio de fase corresponde a todos los posibles
estados de un sistema físico.
Por estado, se entiende la posición y el \textit{momentum}, porque con ellos se
puede con ellos se puede determinar el comportamiento futuro de tal sistema.
La configuración de un espacio puede estar determinada por una \textit{variedad}
$M$ (dimensionalmente finita o infinita) para cada posición $q \in M$ y momentum
$p$, cuyos valores están en el espacio cotangente\footnote{Las velocidades
siempre viven en el espacio tangente $T_q M$, el momentum es definido como $p :=
\partial L  \partial q $, Siendo $L(q, \dot{q}, t)$ el \textit{Lagrangiano} del sistema}  $T_q^*M$. Por lo tanto, el
espacio de estados está naturalmente representado  por
$$
	T^* M := \{ (q, p) : q \in M, p \in T_q^* M \}
$$
y un conjunto de mapas $\Phi_t: M \rightarrow M$, describiendo el cambio en el
instante $t$; $\Phi_t(M)$ es la \textit{fotografía instantanea} en $t$.

El ensemble de mapas $\Phi_t$, sujetos a la evolución temporal

\begin{eqnarray}
	\Phi_0 &=& \text{id}_M \text{(The identity of $M$) } \\
	\Phi_{t+s} &=& \Phi_i o \Phi_s
\end{eqnarray}

wena wena javier 
 
\end{document}


