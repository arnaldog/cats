\documentclass[12pt]{book}

\usepackage[utf8]{inputenc}
\usepackage[spanish]{babel}

\usepackage{amsthm}
\usepackage{amsmath}
\usepackage{amssymb}

%%% MATH COMMANDS SHORCUTS %%%%
\newcommand{\R}{\ensuremath{\mathbb{R}} }
\newcommand{\Z}{\mathbb{Z}}
\newcommand{\N}{\mathbb{N}}
\newcommand{\C}{\mathbb{C}}
\newcommand{\txt}[1]{\text{#1}}

\newcommand{\D}[2]{\frac{d#1}{\d#2}} 

%%% TEXT SHORCUT COMMANDS %%%
\newcommand{\italic}[1]{\textit{#1}}

\newtheorem{mydef}{Definición}
\newtheorem{mytheo}{Teorema}

\title{Propuesta de Tésis Arnaldo Gaspar}

\begin{document}
\maketitle
\tableofcontents

\chapter{Introducción}
% TEORIA DEL CAOS
\section{Time and Chaos principles}
Se puede definir al tiempo como un continuo $t \in \R+$ ó $\R$, donde el
funcional $t = uh(t), u \in
\R$, donde $$\frac{dh(u)}{du} > 0, \quad \forall u \in R$$. El tiempo $t \in
\R_+$ podría especificarse como $\R_+ - \{0\}$ suponiendo como desconocido el
tiempo cero o \textit{original}.

Al respecto, la función $h(u)$ es estrictamente creciente. Al respecto, se puede
considerar a la velocidad del tiempo como la rapidez de cambio de la 
$$
	h(u) = \int_{0}^{u} \frac{df(s)}{dsf(s)}ds = \int_{0}^{u}\frac{df(s)}{f(s)} =  \log f(u)
$$
Si $h(u) = u^2$, entonces 
$$
	e^{u^2} = f(u) 
$$

Por lo tanto, podríamos considerar que el tiempo $t$ correponde a una percepción
logaritmica de un fenómeno exponencial. 

Uno de los conceptos más importantes en la percepción del Universo y la vida es
el del \textit{movimiento}. 

% MOTION AND PHASE STATE
El movimiento está asociado a un conjunto $M \in \R$ llamado \textit{espacio de
fase}.
En la mecánica clásica, el espacio de fase corresponde a todos los posibles
estados de un sistema físico.
Por estado, se entiende la posición y el \textit{momentum}, porque con ellos se
puede con ellos se puede determinar el comportamiento futuro de tal sistema.
La configuración de un espacio puede estar determinada por una \textit{variedad}
$M$ (dimensionalmente finita o infinita) para cada posición $q \in M$ y momentum
$p$, cuyos valores están en el espacio cotangente\footnote{Las velocidades
siempre viven en el espacio tangente $T_q M$, el momentum es definido como $p :=
\partial L  \partial q $, Siendo $L(q, \dot{q}, t)$ el \textit{Lagrangiano} del sistema}  $T_q^*M$. Por lo tanto, el
espacio de estados está naturalmente representado  por
$$
	T^* M := \{ (q, p) : q \in M, p \in T_q^* M \}
$$
y un conjunto de mapas $\Phi_t: M \rightarrow M$, describiendo el cambio en el
instante $t$; $\Phi_t(M)$ es la \textit{fotografía instantanea} en $t$.

El ensemble de mapas $\Phi_t$, sujetos a la evolución temporal

\begin{eqnarray}
	\Phi_0 &=& \text{id}_M \text{(La identidad de $M$) } \\
	\Phi_{t+s} &=& \Phi_t \circ \Phi_s
\end{eqnarray}

En un caso simple, cuando $t$ está medido discretamente en incrementos iguales
de tiempo, un \textit{sistema dinámico} es solamente la secuencia $(f^n)_n$ de
todas las iteraciones de la funcion $f: M \rightarrow M$, por ejemplo:
$$
	f^0 = \text{id}_M, \quad f^n = f \circ \ldots \circ f
$$
\newcommand{\ra}{\rightarrow}
Entonces, se entenderá al sistema dinámico $f: M \rightarrow M$ como un sitema discreto
generado por $f$.  El problema principal que le concierne a esos sistemas es el
siguiente: \textit{dada una función $f$ y un valor inicial $a$, ¿que ocurre con
la secuencia}
$$
	x_0 = a, \quad x_n = f(x_{n-1}) \quad \forall n \in \mathbb{N}^+
$$
\textit{de todas las iteraciones de $f$ computadas en $a$?}

\newcommand{\ti}{\textit}
\newcommand{\bigO}{\mathcal{O}}
La secuencia $(f^n(a))_n$ es llamada \textit{trayectoria} de $a$, y el conjunto
de sus valores, la \ti{orbita} de $a$. Usualmente denotada por $\bigO$

La teoría de las secuencias recurrentes tratan con una sola trayectoria a la
vez, mientras que la teoría de los sistemas dinámicos tratan con un
\ti{ensemble} de todas las trayectorias de un sistema.

Poincaré fue el primero en el estudio global de las iteraciones (y el estudio
cualitativo de los sitemas dinámicos).

% peridic point
Si existe un número $n$ en $\mathbb{N}^+$ tal que $f^n(a) = a$, entonces $a$ es
llamado un \ti{punto periódico} (de período $n$); entonces el \ti{el periodo
principal} de un punto periódico $a$ es el menor $n \in \mathbb{N}^+$ tal que 
$f^n(a) = a$. La orbita de cualquier punto periódico es un conjunto finito. Esto
rduce la orbita a un \ti{singleton} si $a$ es un punto fijo de $f$, por ejemplo 
$$
	f(a) = a
$$

La identidad de \R admite todos los puntos de \R como puntos fijos. El mapa
$f(x) = -x, \quad x \in \R$, tiene un uno punto fijo (el cual es el origen),
todos los otros son periodicos, con período principal 2.

$$
	f(0)=0, f^2(0)=0, f(1) = -1, f^2(1)=1
$$
\paragraph{Conjuntos invariantes} Sea el mapeo $f: M \ra M $, un subconjunto $A
\subset M$, se dice que es invariante si
$$
	f(A) = A
$$

\newcommand{\map}{\ensuremath{f: M \ra M} }
\paragraph{Análisis Gráfico de Pasos} dado un sistema dinámico \map el \ti{análisis
gráfico} de $f$ se realiza de la siguiente manera.
\begin{enumerate}
	\item Dibujar una linea perpendicular al eje $x$ en $x_0 = a$ e
intersectarlo en en el gráfico de $f(x)$ en $f(x_0)$. En el punto $(x_0,
f(x_0))$
	\item Luego una linea paralela en la dirección de $x$ hacia la curva
$x=y$, llegando hasta el punto $(f(x_0), f(x_0))$.
	\item Desde $(f(x_0), f(x_0))$ perpendicularmente a $x$ llegar hasta la
curva, en el punto $(f(x_0), f(f(x_0)) = (f(x_0), f^2(x_0))$ .
	\item Asi consecutivamente.
\end{enumerate}
La intersección de $f$ con $x=y$ determina los puntos fijos.
De acuerdo a donde se comience se puede ver la orientación del avance y también
los pasos muestran la monotonía.

Por ejemplo, el mapa 
$$
	f: [-1, \infty) \ra [-1, \infty), \quad f=\sqrt{1 + x} 
$$

tiene soluciones de $x^2 = x + 1$ con $x = \varphi = (1 + \sqrt{5})/2$ el
que sería un número fijo para $f$.  El análisis gráfico revela la convergencia
de todas las trayectorias hacia $\varphi$; de una forma más precisa, si $x_0 <
\varphi$, entonces $f^n(x_0) \nearrow \varphi$, y si $x_0 > \varphi$, entonces
$f^n(x_0) \searrow \varphi$.


Para el caso de $f: \R \ra \R$, con $f(x) = \lambda x$, el comportamiento cambia
según el valor de $\lambda$.

%%% UN Atractor
\begin{mydef}
Un punto $p$, se dice que es un \ti{atractor} para un sistema dinámico \map si
existe una vecindad $U$ de $p$ tal que
$$
	f^n(x_0) \ra p \quad \forall x_0 \in U
$$ 
\end{mydef}

El conjunto de condiciones iniciales sobre las cuales itera $f$ y converge a $p$
es llamado \ti{dominio de atracción} de $p$.

\begin{mydef}
Un punto $p$ es llamado una fuente o (repeledor) para \map si existe una
vecindad $U$ de $p$, tal que $\forall x_0 \in U, x_0 \not = p$ existe un $n \in
\N$ para el cual $f^n(x_0) \not \in U$
\end{mydef} 

También existen los puntos fijos \ti{indeferentes} que no son atractores ni
repeledores, con el caso del origen respecto al sistema dinámico generado por la
identidad de \R.

\paragraph{Hiperbolicidad} Suponga que \map es un mapa \ti{suave} actuando en
intervalos. Una orbita periódica $\bigO(p)$ de $f$ es \ti{hiperbolica} si
$$
	| (f^m)(p) \not = 1 |
$$
donde obviamente $p$ es un punto periódico y donde $m$ es su período principal.

Si se considera un $k \in \{0, \ldots, m-1\}$ cualquiera, derivando la expresión
se obtiene que 
$$
	(f^m)'(x) = (f^m)'(p) \quad \forall x \in \bigO(p)
$$

Entonces una orbita se aproxima a un conjunto cerrado mediante el uso de la
distancia de $x$ a un conjunto $A$, mediante
$$
	d(x, A) = \inf \{|x-a|; a \in A\}
$$

\begin{mytheo}
Suponga que f es un mapa suave del tipo $C^1$ en el intervalo $M$ y $p$ es un
punto periódico de período principal $m$.
\begin{itemize}
	\item si $|()'() < 1|$, entonces existe una vecindad $U$ de $\bigO(p)$ tal
que $f(\overline{U}) \subset U$, y $\forall x \in U$
$$
	\lim_{n \ra \infty} d(f^n(x), \bigO(p)) = 0
$$
	\item 
\end{itemize}
\end{mytheo}


%\chapter{Teoría del Cáos}
%% EL TIEMPO
%\chapter{Sistemas Dinámicos y Comportamiento Caótico}
%\section{Sistemas dinámicos}
%\section{Conjuntos Invariantes}
%\section{Teorema de Sharkovsky}
%\section{Equivalencia Topológica}
%\section{Estabilidad Estructural}
%\section{El Mapa de Poincaré}
%\section{Dependencia Sensitiva a Condiciones Iniciales}
%\section{Exponentes de Lyapunov}
%\section{Dinámicas Symbólicas}
%\section{Mapa Logistico para $\lambda =4$}
%\section{The Smale Horseshoe}
%\section{Dimensión de Hasdorff y sus Relativos}
%\section{Que es Cáos?}
%
%\chapter{Machine Learning}
%
\chapter{Estado del Arte}
Fuzzy prediction of chaotic time series based on singular value decomposition

Chaotic time series prediction with residual analysis method using hybrid
Elman–NARX neural networks

Chaotic Time Series Forecasting Base on Fuzzy Adaptive PSO for Feedforward Neural Network Training 
Multivariate chaotic time series prediction based on Hierarchic Reservoirs 

Chaotic time series prediction based on phase space reconstruction and LSSVR
model 

Modelling of Chaotic Time Series Using Minimax Probability Machine
Regression

State-Space Reconstruction and Prediction of Chaotic Time Series based on Fuzzy
Clustering

TSK interval type-2 fuzzy neural networks for chaotic time series prediction 

\chapter{Time Series Analysis}
\section{Conceptos Básicos de las series de tiempo caóticas}
Se han definido cantidades invariantes al operador de evolución de un sistema
dimámico y a sus condiciones iniciales. Tales cantidades son tales como la
\ti{dimensión} y los exponentes de \ti{Lyapunov}. Además son independientes del
sistema de coordenadas en el cual se observa el atractor. 
La ventaja de estas cantidades es que otorgan una medida para comparar series de
tiempos diferentes. %citar el paper referencia
Los atractores generalmente están definidos en areas finitas, por lo tanto sus
trayectorias muestran algun tipo de comportamiento oscilatorio (recurrente).

Para mostrar el comportamiento oscilatorio se suele realizar una transformación
bi-dimensional llamada transformación de Hilbert. 

Para obtener el atractor de una serie de tiempo dada, se debe reconstruir su
espacio de fase auxiliar mediante algún procedimiento embebido, que es, la
técnica del tiempo embebido.


\subsection{Dimensiones}
La dimensión cuantifica la auto-similaridad (dimensión fractal) de un objeto
geométrico [7, 8] .

Para un objeto homogeneo, su dimensión es un numero fijo;
para un objeto heterogeneo, sus diferentes partes puedente tener dimensiones
distintas y necesitaría ser caracterizado por una dimension multifractal.

Un atractor caótico suele tener una dimensión fractal. Esta dimensión puede ser
caracterizada de diferentes formas: \ti{counting-box}, \ti{information
dimension} and \ti{correlation dimension}.

\newcommand{\e}{\ensuremath{\epsilon}}
\paragraph{Counting box} $D_0$, se divide la región donde está contenido el
atractor en regiones de radio o arista \e y revisar cuántas veces ($N(\e)$)
cubrir los puntos del atractor. 
$$
	D_0 = \lim_{\e \rightarrow 0} \frac{\ln N(\e)}{\ln(1/\e)}
$$
[AGREGAR EJEMPLOS]

\paragraph{Information Dimension} Una forma abreviada de calcular el \ti{counting box} calcular todas las regiones
en las cuales las trayectorias sean las mismas.

$$
	D_1 = \lim_{\e \rightarrow 0} \frac{\ln H(\e)}{\ln(1/\e)}
$$
donde
$$
	H(\e) = - \sum_{i=1}^{N(\e)} P_i \ln P_i
$$
$P_i$ es la frecuencia relativa con la cual una trayectoria típica entra en la
caja $i$ de cobertura.
[AGREGAR EJEMPLOS]

\paragraph{Correlation Dimension} Las dimensiones $D_0$ y $D_1$ son costosas de calcular (en tiempo). 
La dimensión de correlacion por el contrario es mucho más facil de calcular, se
define como
$$
	D_2 = \lim_{\e \rightarrow 0} \frac{\ln \sum_{i=1}^{N(\e)} P_i^2}{\ln \e }
$$
donde, el numerador puede ser expresado como [CITAR]
$$
\ln \sum_{i=1}^{N(\e)} P_i^2 = C(\epsilon) =
\frac{2}{N(N-1)}\sum_{j=1}^{N}\sum_{i=1}^{N} \Theta(\e - |x_i - x_j|)
$$

donde $\Theta(\cdot)$ es la función escalón unitario, $\Theta(x) = 1$, para $x
\geq 0$ y $0$ en el caso contrario y $|x_i -x_j|$ entre puntos. Con esto, la
expresión queda definida como:
$$
	D_2 = \lim_{\e \rightarrow 0} \frac{\ln C(\e)}{\ln \e}
$$

La dimensión de correlación puede ser calculada en dos pasos. 
Determinar $C(\e)$ para varios $m$ [REVISAR]

\subsection{Exponentes de Lyapunov}
Los exponentes de Lyapunov son la cantidad más importante para determinar si un
sitema dinámico es caótico.  Un exponente de Lyapunov positivo máximo es una
fuerte señal de existencia de caos. [CASO CONTRARIO]
Un sistema $m$ dimensional tiene $m$ exponentes de Lyapunov $\lambda_1,
\lambda_2, \ldots, \lambda_m$ en orden descendiente.

$$
	\frac{d\delta x}{dt} = \frac{ \partial F }{ \partial x}  \delta x
$$
lo cual genera
$$
	\delta x(t) = A^t \delta x(0)
$$
donde 
$$
	A^t = \exp \{ \textstyle \int (\partial F/\partial x)dt \}
$$
es un operador lineal que evoluciona a un vector infinitesimal en el tiempo $0$
al tiempo $t$. La tasa exponencial promedio de divergencia al vector tangente
esta dado por
$$
	\lambda [ x(x), \delta x(0) ] = \lim_{t \rightarrow \infty} \frac{1}{t}
\ln  \left\lvert \frac{\delta x(t)}{\delta x(0)} \right\rvert
$$
Los exponentes de Lyapunov están dados por
$$
	\lambda_i \equiv \lambda[ x(0), e_i]
$$

donde $e_i$ es un vector base $m$-dimensional. La interpretación de los
exponentes de Lyapunov es la de la media de las tasas de divergencias locales
sobre el atractor. Los valores $\lambda_i$ no dependen de las condiciones
iniciales debido a ergodicidad [CITAR O EXPLICAR].

Para probar si una serie de tiempos es caótica o no, se necesita calcular al
menos el máximo exponente de Lyapunov $\lambda_i = \lambda_{\text{max}}$

\paragraph{Calculo del exponente de Lyapunov} Para calcular
$\lambda_{\text{max}}$ se debe hacer lo siguiente [CITAR]
\begin{itemize}
	\item Elegir dos puntos iniciales cercanos, cuya distancia $d_0 \ll 1$
	\item Integrar el sitema dinámico para un intervalo de tiempo pequeño
$\tau$, cuyos pasos tendrá distancia $d_i$.
\end{itemize}
$$
	\lambda_{\max}=\lim_{n\ra\infty} \frac{1}{n\tau}\sum_{i=1}^{n}
\ln\frac{d_i}{d_0}
$$
Para sistemas dinámicos discretos
$$
	\lambda_{\max} = \lim_{N\ra\infty} \frac{1}{N}
\sum_{n=1}^{N}\ln\frac{\|dx_n\|}{\|dx_{n-1}\|}
$$
donde $\|dx_n\| = \sqrt{\sum_{i=1}^{m}[dx_n(i)]^2}$

En la práctica, las series de tiepmo no tenen todos sus puntos equidistantes en
$t=n\tau$. Por lo cual se puede realizar mediante comparaciones en el tiempo
[COMPLETAR]
$$
	\lambda_{\max} =
\lim_{N\ra\infty}\frac{1}{t_N-t_0}\sum_{k=1}^{N}\ln\frac{d_1(t_k)}{d_0(t_{k-1})}
$$

La data experimental generalmente esta contaminada con ruido[EXPLICAR]. Su
influencia puede ser minimizada usando estadisticos de promedio para calcular
los exponentes de Lyapunov [PROFUNDIZAR].


\subsection{Transformada de Hilbert y Transformada de Fourier y Wavelet}
Análisis espectral para estudiar la periodicidad de la serie de tiempos. 

[ESCRIBIR SOBRE ESTO]

\subsection{Reconstrucción del Atractor}
[TAKENS THEOREM]
La reconstrucción del espacio de fase es la tarea más importante en el análisis
de series de tiempo caóticas.
En el caso de las series temporales, la tarea significa descifrar la
trayectoria porque de ellas usualmente se obtiene un muestreo escalar de algunas
variables solamente.
Takens[12], desarrolló un teorema que perimite reconstruir el sistema dinámico
de una serie de tiempo.

El método de Takens, significa muestrear una serie de tiempo $s(k) = s(t_0 +
k\Delta t)$ con $\Delta t$ como intervalo de muestreo, de la siguiente manera
$$
	u(t) = \{s(t), s(t + \tau), \ldots, s(t + (m-1)\tau) \}
$$

donde $t=t_0 + k\Delta t$, y $\tau$ es el tiempo de retardo que es múltiplo de
$\Delta t$ y $m$ es la dimensión embebida del sistema. 
La determinación de $\tau$ y $m$ son los desafios centrales en la reconstrucción
del atractor de la serie temporal basado en la técnica de Takens. 

\subsubsection{Dimensión embebida}
La dimensión embebida del sistema puede ser de un alto grado.
El teorema de Takens provee una cota inferior para $m$.
En un espacio de dimensión $m$, hay un subespacio de dimensión $d_1$ y $d_2$,
genéricamente intersectan en subespacios de dimensión [AGREGAR MAS].

\subsubsection{Time Delay}
El teorema de la dimensión embebida establece la independencia entre la
dimensión embebida $m$ y el tiempo de retardo de muestreo (time delay). 
El valor de $\tau$ debe satistfacer las siguientes condiciones:
\begin{enumerate}
	\item $\tau \varpropto \tau_s$, donde $\tau_s$ es el período de
muestreo.
	\item $\tau$ no debe ser muy corto, de lo contrario $P(s(t)) \cap
P(s(t+\tau)) \not = P(s(t))P(s(t+\tau))$ no hay independencia entre las
variables.
	\item Si $\tau$ es muy grande, el sitema es intrinsecamente muy
inestable y las diferencias pueden ser aleatorias respecto a las otras.
\end{enumerate}

La diferencia entre $s(t)$ y $s(t+\tau)$ debe ser tal que no sean
estadisticamente independientes ni deterministicamente dependientes.

Es un problema muy complicado, que puede ser resuelto usando el algoritmo de
elección de información mínima, que proviene de la teória de la información
[CITAR].

\subsubsection{Información Mutua $I(\tau)$}
Para datos muestreados $s(t)$, se crea un histograma para observar la
distribución de la data. Sea $p_i$ la probabilidad empírica que la señal asume
en el valor de la $i$-ésima clase, y sea $p(\tau)$ la probabilidad de que
$s(t)$ esté en la $i$-ésima clase y además que $s(t+\tau)$ esté en la $j$-ésima
clase. Entonces la \ti{información mutua} para un retardo de tiempo $\tau$ es:
$$
	I(\tau) = \sum_{i,j} p_{ij}(\tau) \ln p_{ij} (\tau) - 2 \sum_i p_i \ln
p_i
$$
Cuando $\tau=0$, entonces $p_{ij} = p_i \delta_{ij}$, lo cual devela la entropía
de Shannon de la distribución. 

% Es facil de calcular y obtener
Algunos autores [Franser and Swinney][19], sugieren que $I(\tau)$, es una
especiede función de autocorrelación no lineal para determinar el óptimo $\tau$. 
Lo frecuente es usar el primer mínimo de $I(\tau)$ como marka para obtener el
tiempo de retardo óptimo para reconstruir el espacio de fase.

[EJEMPLO] El atractor de Lorenz tiene un $\tau_{\min}=10$



\section{Modelado y Pronóstico}



\end{document}



